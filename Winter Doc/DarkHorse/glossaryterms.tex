\begin{document}
\begin{list}
  \item  \textbf{ADA:} Americans with Disabilities Act; one of America's most comprehensive pieces of civil rights legislation that prohibits discrimination against and guarantees people with disabilities have the same opportunities as everyone else to participate in the mainstream of American life.
  \item  \textbf{Assistive Technology:} Assistive, adaptive, and rehabilitative devices for people with disabilities; promotes greater independence by enabling people to perform tasks that they were formerly unable to accomplish, or had great difficulty accomplishing.
  \item  \textbf{Benchmarking:} A standard by which something can be measured or judged.
  \item  \textbf{CEP:} Otherwise known as a Critical Experience Prototype, this is a physical prototype created to make an experience ��real enough�� to gather insights and understanding about the user’s experience.
  \item  \textbf{CFP:} Otherwise known as a Critical Function Prototype, this is a physical prototype built to test a concept that is critical to addressing the problem statement.
  \item \textbf{Control:} The power to influence or direct either people's behavior or the course of events.
  \item \textbf{Dark Horse Prototype:} A device created during the winter quarter of ME310 that was ruled out in the fall quarter or undiscovered due to being too risky or too difficult to complete��; emphasizes creative out-of-the-box thinking and exploring all of the design space for the project. 
  \item \textbf{Disability:} A physical or mental condition that limits a person's movements, senses, or activities.
  \item \textbf{FAA:} Federal Aviation Administration; United States national aviation authority whose mission is to provide the safest, most efficient aerospace system in the world, oversees all aspects of American civil aviation.
  \item \textbf{Herrmann Brain Dominance Instrument (HDBI):} Illustrates and explains the way a person prefers to think, learn, communicate and make decisions. It identifies the preferred approach to emotional, analytical, structural, and strategic thinking.
  \item \textbf{Independence:} Freedom from outside control or support.
  \item \textbf{Limited Mobility:} Mobility impairment may be caused by a number of factors, such as disease, an accident, or a congenital disorder and may be the result from neuro-muscular or orthopedic impairments. It may include conditions such as spinal cord injury, paralysis, muscular dystrophy and cerebral palsy. It may be combined with other problems as well (i.e. brain injury, learning disability, hearing or visual impairment).
  \item \textbf{Needfinding:} Discovering opportunities by recognizing the gaps in the system or the needs.
  \item \textbf{Non-Discriminatory:} Fairness in treating people without prejudice.
  \item \textbf{Pain Points:} A level of difficulty sufficient to motivate someone to seek a solution or an alternative; a problem or difficulty.
  \item \textbf{Perspective:} A particular attitude toward or way of regarding something; a point of view.
  \item \textbf{Self-Image:} The idea one has of one's abilities, appearance, and personality
  \item \textbf{ANAC:} Agencia Nacional de Aviação Civil – Brazilian National Agency of Civil Aviation
  \item \textbf{Libras:} Brazilian Sign Language
\end{list}
\end{document}