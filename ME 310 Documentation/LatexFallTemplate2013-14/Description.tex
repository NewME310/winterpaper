\chapter{Design Vision}
%\chapter{Design Description}   %This title will make more sense in Winter and Spring.
\label{design-description}

This \textbf{Design Description} chapter defines what the design is. But for Fall, it's really your vision or proposal for what you think the design might be by the end of Winter or Spring quarter. So let's call it the \textbf{Design Vision} chapter in Fall. It will be a short
chapter. 

Even so, on the basis of preliminary need-finding, benchmarking and critical function evaluation, you should have some idea of what may be appropriate. Take a point of view and assert it!  A CAD rendering, a systems diagram or even a sketch of a concept could help to explain your vision. 

If you find yourself adding rationale, or discussing design alternatives or how the vision came about, you are writing text that should be moved into the end of the Development section. This is section is about what the design (or vision) 
is, not how it came to be.

\begin{remark}\color{blue}
\section{Vision}
\label{vision}

For Winter and Spring, although you now have a design to describe, you probably want to start off your Design Description chapter with a reminder of the overall vision, toward
which your current design is an important intermediate step.

Use this section to describe your vision or proposal for what you think the design might be. Ideally you should have a sketch, a diagram or other images to help define it.  
\section{Specifications}
\label{specifications}

In Winter and Spring here is where you describe what your design is. 
\begin{itemize}
\item Define the subsystems and how they go together.
\item Use diagrams, CAD renderings, tables, etc. to make it clear. 
\item Use flow charts or pseudocode to describe procedures.
\item Detailed source code and numerical data should go in an Appendix section or, if really long, on a CD or other electronic format (nobody will wade through the printout).
\end{itemize}

\normalcolor\end{remark}





