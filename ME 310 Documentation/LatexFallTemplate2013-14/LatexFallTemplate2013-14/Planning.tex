\chapter{Planning}
\label{project-planning}

\begin{remark}\color{blue}
Teams with global partners face special challenges in  terms of organization, project management and planning.
It is a truism that organizational burden goes as the square of team size. 

To address these issues, we ask each local+global team to prepare a \textbf{plan for Winter quarter} to include in this section. You have just accomplished a first, rough critical function prototype (CFP and CEP) and you have given a presentation and written a document that captures the current state of your vision and findings. You have learned who can do what and how much work it really takes. And you are highly motivated to make Winter go more smoothly and to ``take control'' of your project.
\normalcolor \end{remark}

\section{Deliverables}
Define briefly what will be delivered. Of course, this is to the best of your current knowledge -- things will change. A short table with some explanatory text could be used here. Your project plan should include the following non-negotiable items and any sub-tasks or intermediate items'' that lead up to them. Consult the course calendar for dates and more details.

\begin{itemize} \tightlist
\item Travel: When are you traveling? What is planned for the trip?
\item Dark Horse prototype -- a 2nd CFP that probes the edge of the design space
\item Travel Docs due
\item Funky Prototype -- an initial system where a potential avenue for the final product is developed
\item Functional System Review -- your latest and greatest as Winter quarter draws to a close. It should give a clear indication of what to confidently expect in June.
\item Winter Design Documents
\end{itemize}

\section{Milestones}
When are various elements (e.g., rough prototypes, final prototypes) delivered? When are key tests conducted? These are the dates, times, and places where project progress is observable and/or demonstrated. Again, update with planned versus actual dates as the design progresses.

\section{Distributed Team Management}
Explain how your distributed and interdisciplinary team will collaborate, communicate and keep itself on-track with respect to the afore-mentioned deliverables.

\section{Project Budget}
As with any serious proposal, you should include an estimated budget with some specifics about money that has been spent (Fall) and probably will be spent (Winter). Details on vendors can be put in the Appendix. A common mistake is that teams spend too little money until late in the quarter and then spend too much, doing rush jobs and rework.

\section{Project Time Line}
Summarize the projected project time line if it is not already explicit in the project planning representations above.

Use any of the familiar project development representations including lists, Gantt Charts, Pert Charts (Figure \ref{fig:full-page-example}), bubble diagrams, tables, etc. In addition, you will almost certainly need a list or table of items that says a bit
more about the items and gives an idea who is going to do what.

\begin{figure}[bhtp] 
\centering
		\includegraphics[width=\textwidth]{Figures/Ch6/su-tmit-after.pdf}
	\caption[Project task replanning example]{In this example from \cite{Toyota01}, Stanford students collaborated with a group at TMIT, Japan. At the end of the Winter quarter it was decided to abandon one branch of the TMIT effort and to eliminate some of the tight coupling that was originally envisioned. }
	\label{fig:su-tmit}  %Tag for referring to figure in text.
\end{figure}

\begin{figure}[p]   % p for "page" for let it be a full page figure!
\centering
		\includegraphics[angle=90, height= 8in]{Figures/Ch6/ideastorm}
	\caption[Rotated landscape figure example]{An example of taking a large figure and having Latex rotate it 90 degrees to display it in landscape format as a full page figure.}
	\label{fig:full-page-example}  %Tag for referring to figure in text.
\end{figure}

\section{Reflections and Goals}
This is the one section that you would not find in normal research or engineering proposal. But in the spirit that we're doing this in an academic setting, we want to be sure that we reflect on what we're learning and thinking and where we hope to go with it. The reflections are personal, and can be written diplomatically, for example using the \emph{I like..., I wish...} protocol.

A part of reflective section may include how your team functioned in the fall - explaining how and why your actual design process deviated from what you originally planned, if relevant. (Time lines and milestones often have the look of having been concocted the night before the report is due.)
