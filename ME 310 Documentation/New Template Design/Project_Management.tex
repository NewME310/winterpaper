\chapter{Project Management}
\label{Project_Management}

\section{Winter Quarter Review}
Throughout the winter, we and our global team tried to explore our huge design space and identify which were the directions we should pursue and which we should eliminate. To do so we always tried to coordinate the work done at Stanford with the work done in Brazil. \\

At the beginning of the quarter we wanted their project to explore one aspect of the flight experience while we were exploring something else. The idea was then to see which approaches were feasible and which were not. For instance changing the changing the cabin layout turns out to be impossible under the constraint of keeping the number of passengers onboard the same. But at some point, USP and us needed to converge that is what we did after coming back to some of our users’ interviews and need findings. \\

We decided to create a new flight experience for disabled people that would start from the jet way where wheelchair passengers leave their wheelchair. At this point Stanford team decided to provide them with a protection system for their wheelchair as well as a wheelchair tracking device to reassure them and give them piece of mind. \\

From the jet way, once the disabled person is in the wheelchair then he needs to transfer to his seat and to do so without the help of any flight attendant USP decided to focus on the transfer mechanism that would give our passengers more independence. \\

Integrating both our prototype with USP’s, we were able to build our vision for next quarter. When a wheelchair user is without his wheelchair he worries about two things: what happens to his wheelchair, his legs, and how he can actually move without it. By merging our ideas with our global partners’ we should be able next quarter to build a prototype that should this issue.


\section{Deliverables}
The official deliverables for the winter quarter focus on three main prototypes: dark horse, funky, and functional.  In addition to these three prototypes, our team has set more deliverables within each one.  Each prototype will have a dedicated brainstorming session after a project briefing, and two iterations of each prototype must be performed. The dark horse prototype is where we plan to address the futuristic cabin approach and bring that idea to life.  We want to have the basis of our final solution started by the end of dark horse and continually increasing to the end of winter quarter.  Our main personal deliverable will be to have a final solution design space and all the integral components decided before the end of winter so spring will be iteration after iteration until perfection. 

\section{Milestones}
The milestones for the winter quarter are the darkhorse, funky, and functional prototypes.  Each of these will represent a challenge to our design thinking and our solution space.  In addition, we have the milestone of integrating two new team members into the ME310 experience and making a smooth transition.  Another milestone will be the creation of our final design solution space.  We will also be traveling to Brazil at the end of the winter quarter to do a prototype meeting. 

\section{Distributed team management}
The entire team will be working on each of these projects and doing the documentation that accompanies each one.  Each team segment will be responsible for the documentation concerning their prototypes and clearly articulating the lessons learned and the takeaways. We will also distribute the workload more thoroughly next quarter with the addition of two new team members.  We plan to write the documentation in paragraph and bullet form from the beginning next quarter instead of bullets to make an easier transition.  The bullet format will still be used to distribute ideas to our global team and the teaching team.  

As for each team member’s role for the Stanford team, it has not been decided who will be the chief documentation and chief financial officers or if all the roles will be switched to accommodate the new team dynamic. 

\section{Project Budget}
Below is the budget planning for the winter quarter prototypes.  Iterations of designs are included in our budget to encourage testing and learning from failures. The budget is on the overzealous side and we hope to spend less than the allotted amount so that the majority of the money can be used during spring for the iterations and the final cabin experience.
\subsection{Stanford Budget}
\subsection{USB Budget}
 

\section{Project Time Line}
\section{Reflection and Goals}
\subsection{Rodrigo}
So far I’ve learned a lot of new methods and techniques of doing a project. The "always prototype" methodology is a great way to visualize our ideas and learn from them as soon as possible.
I enjoyed the interaction we had with our team at Stanford, it is great to interact and work with people of different background and culture. Even though we have a six hours time zone difference we were able to get synchronized quite well and our communication and file sharing was pretty effective.
I believe that the in next quarters we'll learn a lot more, because new prototypes are going to be made and with the experience acquired from them we'll become closer to our solution, since we'll have a better vision of our problem.

\subsection{Luiz}
On this initial quarter I’ve learned a lot of things, not only new techniques but also new ways to work and deal with different cultures. I’ve learned the importance of early prototyping, which makes the “relationship” with the idea easier and anticipates errors, contributing thus to learning faster and reducing the total costs.

I’ve also learned the importance of empathy especially when the final user is different from us, it is important to get to know the feelings of the people in order to make a good design; and the importance of setting up a good communication platform, it’s hard to conciliate different time zones and cultures. Another important thing I’ve consolidated after this quarter is the power of working in group. It’s true that we had some troubles, but working as group we get over it.

I believe that the next quarters are full with great new discoveries especially when you have in mind we still got to work on 3 or 4 prototypes. I believe I’ll learn a lot about users and their needs, and with that, grow as person by being able to understand others.

\subsection{Amanda}
So far, my learning encompassed different areas. First, I was able to deepen my understanding of a new Design methodology, making the comparison with other methods to enrich the design choices possible. This knowledge can and will be use during my professional life. Furthermore, I’ve learned different techniques for each stage of the project such as the execution of personas.

The fact of having a project with a large scope allowed me to better understand the life and situation of people with different disabilities. This understanding has created a new look at the restricted environment of the airplane, which made us work with a great number of contradictions. This kind of situation makes us overcome big challenges.

Furthermore the understanding of the regulations and the operation of the aircraft system was also a great learning experience.

In relation to group communication, the cultural differences, language, time zone and distance are factors that contributed to making this experience unique.    

\subsection{Guilherme}
While the open-ended nature of the problem was intimidating at first, I think we have done good job in handling it and fractioning it down to specific activities instead of particular groups of people. This was particularly a great experience because it was one of the first times that I received a vague problem like this one and had to first find out what the actual problem was so that I could try to find a solution for it. I was also impressed with the learning curve that making a prototype provides. This is shocking because in four years of an engineering college, I’ve only made a handful of prototypes. Therefore I believe this has been a major discovery.

Another take away that I noticed is that being in a diverse group contributes positively to the creativity of the group.  I have made several projects with Industrial Engineers, but never once I saw as many new ideas as in this project. For instance, Amanda who studies Product Design, surprised me positively with her prototyping ideas, including the “Claymotion” one. The simplicity and effectiveness of the Swivel Chair designed by Erika and Maria also impressed me.

In spite of the communication problems we had on the beginning of the project, I believe that our team has bonded the past few weeks. We did have some conflicts throughout the semester, but I believe they are natural and sometimes welcome, because it allows us to improve. On a personal note, I believe I’ve failed to consistently check the courses calendar and may have upset Maria and Erika a few times because of the lack of planning from our part. This issue should not happen again because our misbelief that we did not have to follow ME310’s calendar, expect for the final deliveries, was already corrected. On the other hand, I believe we did a quality job elaborating both the presentation and documentation and I’m looking forward to see our final prototype.

Looking forward to meeting our new teammates and buying them a pint (at Podio, or later on in Sao Paulo)!

\subsection{Erika}
So far this quarter I have learned a great deal in both a personal and design perspective.  I have learned about my shortcomings and weaknesses and where I need improvement before I enter the workforce.  But most importantly, I learned how to bring my personality to a design challenge such as the one we are being presented with.  I have never been in a design course of this vigor or intensity.  It has been a very eye-opening experience to all the work that needs to happen to define a problem more specifically when given an open-ended problem.  The needfinding and benchmarking was very informative and showed me the lack of knowledge I have had within the world of disabilities and how they are treated.  In addition, I learned the importance of team dynamics and its effect on making a successful and healthy team.  My passion for design has grown from this course, and I am excited for the next two quarters.  I cannot wait to see our final design solution and to help people around the world do something they love (TRAVEL!!!). 

\subsection{Maria}
This quarter has been an incredible learning experience. Our team was lucky in that we got a project we were extremely passionate about and were able to recruit a team of wonderful advisors and invested parties. This project promotes so much passion within our interviewees and it has been great to learn about the daily struggles of disabled people because we tend to take all of these “normal” things for granted. I’ve loved working with the USP team, I have witnessed the evolution from the lost duckling to the roaring lion, at first confused as to how to proceed but now being leaders and calling the shots. It’s wonderful. Obviously our team has gone through a lot but I am really excited for what’s to come. The process of finding potential teammates that would be as passionate about the project as we are and would also contribute both academically and emotionally to the team was a very interesting one. I actually think every team should be forced to undergo this process because it really makes you think about what truly matters when building a team. Additionally, bringing in people from other disciplines will be extremely valuable, as they will come with their own experiences and biases to complement ours. I am excited to see what amazing things we will come up with this upcoming quarters and even more excited for the journey that takes us there (and Brazil, duh!).

\subsection{Laura}
As a new member of the team it was a little hard for me to jump right into dark horse as a first prototype to build, but once I understood what were the expectations both from the class and from our users I found at that this project is a great opportunity to merge my engineering skills (I’m from an aerospace engineering background) with empathy and understanding of the needs and issues of people you are designing for. \\

I know a lot of aircraft design because this is the field where I’d like to start my career, but anytime I was taught about it, it was mainly in terms of aircraft performance and never in terms of people’s need. I think this project is allows to see aircraft design from the user’s point of view instead of the aircraft manufacturer’s point of view and for me this is extremely valuable. \\

Since I come from France it is actually the first time that I have the opportunity to work with both American and Brazilian people and this cultural diversity is also a great source of enrichment. I’m looking forward to go to Brazil to meet our global partners in person. \\

But beyond this, this project now means a lot to me. I got the opportunity to talk to disabled people and understood how painful it is for them to travel. They care a lot about our project because they see it as way to improve their experience and I don’t want to disappoint them. They deserve the right to enjoy their flights the way we do and for this reason I’m very excited about Spring Quarter and I can’t wait to build our final prototype and test it.