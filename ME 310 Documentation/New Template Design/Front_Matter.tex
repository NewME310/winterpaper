%The executive summary is a special chapter before the TOC
% See the other sections (e.g. Context) for more normal chapter headings
%%%%%%%%%%%%%%%%%%%%%%%%%%%%%%%%%%%%
% Call it Front Matter in TOC, as it will include Glossary and auto-generated TOC, LOF.
\chapter[Front Matter]{Front Matter}
\label{cha:front}
\addcontentsline{toc}{section}{Executive Summary}
%%%%%%%%%%%%%%%%%%%%%%%%%%%%%%%%%%%%
%Begin the actual executive summary text. If you create any subsections you
%probably want to use  \section*{Section name}  with an asterisk, so they are not numbered.
% Note: to get proper looking quotes use two left/right single quotes: ``. . . ''

%Example of a remark that can be optionally printed:In a world that is dynamic in almost every aspect, mobility is a necessity.  However, for persons that suffer from disabilities, their mobility may be limited.  Traveling with limited mobility can be a very difficult and burdening task especially if the travelling involves being within the small confines of an airplane.  Passengers that are faced with limited mobility or a disability have a different and often times worse experience than the average passenger during the entire flight experience.  How can have all passengers have the same experience? Designing a new futuristic cabin and creating a new travel system are essential for achieving the same experience for all passengers and to give passengers with limited mobility independence and control.

Embraer, the Brazilian airline manufacturer, decided to partner with Stanford University and the University of Sao Paulo to solve the problems inherent in the air travel experience for persons with limited mobility or disabilities. The team from Stanford University is composed of five students and the university of Sao Paulo team is composed of four students.  In collaboration, we started this journey toward a solution through extensive needfinding and benchmarking.  The needfinding centered on conducting user interviews for both the disabled passenger and the flight crew while benchmarking focused on analogous situations, patents, regulations, and current concepts and solutions.

We've spent the last two quarters in a continuous cycle of needfinding, prototyping, testing, and learning and validation. This process began in the fall with a number of user interviews, which led the development of a critical function prototype and a critical experience prototype, one being a more incremental fix while the other addressed a more futuristic cabin.  The incremental fix was a swivel chair that would address the window versus aisle debate in the Embraer cabins with rows of 2 seats.  But what if we wanted to apply our solution to larger cabins? We then looked at a more futuristic design, which came in the CFP of seats of rails. 

Over the course of the winter quarter we've gone through several more prototyping cycles as well as additional benchmarking, user interviews, surveys, and testing. The result of this extensive research is our collective focus on users' independence, both on and off the plane. To accomplish this goal we've proposed two subsystems. One is cabin-focused, including a redesigned aisle transfer system, allowing for a smoother transition for the passenger with less reliance on the airline employees, and an enhanced environmental control system which also provides increased autonomy to the passenger. The other component is a device and system for storing and protecting the passenger's wheelchair or mobility device from the moment they leave it behind on the jetway until they arrive at their destination, providing peace of mind along the way. 

The key to independence for these passengers is not only an improved flying experience but also the knowledge that, upon arrival at their destination, they can go on with their lives without the major disruption of a damaged mobility device.

%\section*{Glossary}

%\begin{list}{-}{}

%\end{list}

% Set up the Glossary. The template is looking for a file called
% "glossaryterms.tex" with glossary terms and definitions.
% You can either edit this file manually or you
% can use the Memoir glossary feature in which you insert items like
%   \glossary{glossary term}{our definition of what the term means}
% wherever you like, as you write your documentation.
% When you run the report through Latex, it will create a ".glo" file like
% "OurFallDocument.glo" which you can edit to create the file "glossaryterms.tex"
% There is also perl script I made which will do the formatting for you. 
%  perl Glo2Tex OurFallDocument.glo > glossaryterms.tex

\newpage
\section*{Glossary}
\addcontentsline{toc}{section}{Glossary}
\label{sec-glossary}
\begin{description}
\item [3d audio technology] Simulation that creates the illusion of sound sources placed anywhere in 3 dimensional space, including behind, above or below the listener.
\item[action-event control] Process where a user action creates an physical event.
\item [API] Application Programming Interface.
\item[array of microphones] Microphones linked together to expand the effective coverage area. 
\item [Ausim] 3D audio hardware company.
\item [Automatic beam steering] Signal processing technique to narrow the microphone coverage area. Used to pick out a speaker and suppress background noise coming from directions other than that of the speaker.
\item[Benchmarking] A process of researching and observing to understand the state of the art for a given field or topic.
\item[Brainstorming] A process by which groups of people generate ideas
\item [Brainwaves] A common term that refers to post-synaptic potentials measured from many neurons in the brain
\item [CDR] Center for Design Research at Stanford University
\item [CFP] Otherwise known as a Critical Function Prototype, this is a prototype built to test a concept that is critical to addressing the problem statement.
\item [Client] Computer program that accesses a server.
\item [Client-server paradigm] A computing architecture which separates the client from a server over a computer network. 
\item [Crowded channel] A communication channel that is clogged with information.
\item [CVE] Acronym for Collaborative Virtual Environment. This is a virtual environment that support more than one user at the same time.
\item [Dark Horse] An idea that is unlike the others preceding it, an outlier.
   % input the list "glossaryterms.tex"
\end{description}

%%%%%% Example of an optionally printed "remark"
%\begin{remark}
%\color{blue}
%It's a sign of a successful team that the glossary becomes extensive. Define any non-obvious or invented terms. For %example, if you reference something by an acronym, that might be a glossary term. Teams also coin terms to describe %design features. Define such terms here.  Don't define obvious stuff (axle, keyboard).  

%See comments in me310report.tex if you want to generate a glossary semi-automatically from tagged keywords.
%\normalcolor
%\end{remark}

%%%%%%%%%%%%%%%%%%%%%%%%
% TOC and LOF are automatically generated -- Note that sometimes have to "compile" Latex THREE
% times to update the main .aux files, the TOC etc. files, and finally the PDF output with all changes
% propagated to the printout.
% Make Table of Contents title smaller than a normal Chapter heading:
\renewcommand{\chaptitlefont}{\normalfont\Large\bfseries}
\newpage
\tableofcontents %asterisk to prevent it from getting a number

% Optional Lists of Figures and Tables:
\newpage
\listoffigures  %Note that for this you probably want to add the [short-headings] to captions.
%\listoftables  %I decided to omit the LOT in this example.

%Back to normal size for subsequent sections
\renewcommand{\chaptitlefont}{\normalfont\Huge\bfseries}
%%%%%%%%%%%%%%%%%%%%%%%%

