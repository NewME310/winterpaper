%The executive summary is a special chapter before the TOC
% See the other sections (e.g. Context) for more normal chapter headings
%%%%%%%%%%%%%%%%%%%%%%%%%%%%%%%%%%%%
% Call it Front Matter in TOC, as it will include Glossary and auto-generated TOC, LOF.
\chapter[Front Matter]{Front Matter}
\label{cha:front}
\addcontentsline{toc}{section}{Executive Summary}
%%%%%%%%%%%%%%%%%%%%%%%%%%%%%%%%%%%%
%Begin the actual executive summary text. If you create any subsections you
%probably want to use  \section*{Section name}  with an asterisk, so they are not numbered.
% Note: to get proper looking quotes use two left/right single quotes: ``. . . ''

%Example of a remark that can be optionally printed:In a world that is dynamic in almost every aspect, mobility is a necessity.  However, for persons that suffer from disabilities, their mobility may be limited.  Traveling with limited mobility can be a very difficult and burdening task especially if the travelling involves being within the small confines of an airplane.  Passengers that are faced with limited mobility or a disability have a different and often times worse experience than the average passenger during the entire flight experience.  How can have all passengers have the same experience? Designing a new futuristic cabin and creating a new travel system are essential for achieving the same experience for all passengers and to give passengers with limited mobility independence and control.

Embraer, the Brazilian airline manufacturer, decided to partner with Stanford University and the University of Sao Paulo to approach this problem of improving the entire air travel experience for persons with limited mobility or disabilities. The team from Stanford University is composed of two students and the university of Sao Paulo team is composed of 4 students, all with engineering degrees.  In collaboration, we started this journey toward a solution through extensive needfinding and benchmarking.  The needfinding centered on conducting user interviews for both the disabled passenger and the flight crew while benchmarking focused on analogous situations, patents, regulations, and current concepts and solutions.

The research that was conducted during needfinding and benchmarking was instrumental in the approach we are taking toward a solution.  The user interviews led us to the five themes we need to address with our future solution.  These themes are customer service, control, independence, seat preferences, and non-discriminatory. Figure 1 1 shows the themes and how they each rely on the others to be successful.  The interviews with potential users revealed horror stories that dealt with customer service or the lack thereof.  The solution space needs to create an environment that limits the interaction between the flight crew and the passenger to prevent these horror stories from becoming a reality for future travelers.   Independence and control were also instrumental in our findings.  The users of our solution want to feel independent and in control of their situation even though they might need assistance.  This leads our solution path to one that centers on automation and allowing the user to control their surroundings instead of the other way around.  One major discovery we made concerned the seat that a person with limited mobility chose when boarding the flight.  They chose to sit in the window seat instead of the easier-to-access aisle seat to accommodate other passengers, not themselves.  This brought us to the idea of making every seat accessible for all passengers regardless of mobility status.  The final theme motivating our solution is a non-discriminatory design.  Limited mobility passengers and passengers with disabilities have a condition that singles them out to begin with so why should our design add insult to injury by singling them out more?  Therefore, we are focusing on a universal design that would aid and improve the experience for both the limited mobility passenger and the average passenger.

These themes were our driving forces for the critical function and critical experience prototypes we created to further explore our problem space.  The team created a number of prototypes but really focused on the ones that solved this problem; one being a more incremental fix while the other addressed a more futuristic cabin.  The incremental fix was a swivel chair that would address the window versus aisle debate in the Embraer cabins with rows of 2 seats.  But what if we wanted to apply our solution to larger cabins? We then looked at a more futuristic design, which came in the CFP of seats of rails. Figure 1 2 shows the concept of the seats on rails in a clay mock-up. Here, the rows will move forward and back to provide a certain row with extra room to allow a passenger to get in and out without disturbing the other passengers.  This concept brought light to all the solutions that could be implemented and what we could make the design space to be.

\begin{figure}[h]
  \centering
     \includegraphics[width=7cm]{images/image007.png}
   \caption{Main themes driving our solution}
  \label{fig:main_themes}
\end{figure}

Our vision for a solution is a more dynamic cabin that allows the user to customize the space to his needs and allows for a more enjoyable and interactive experience.  If the world we live in is dynamic, then why does the airplane cabin have to be static with the same seating arrangements in all planes?  This is what we want to change.  We want to change the way a passenger looks at the flying experience and how they feel before, during, and after the flight.  The passengers should have more control over the seat selection, the firmness/softness of their seat, the angle, and the orientation; this list is endless. Giving passengers more independence and control while minimizing customer service interaction and discrimination is our motivation for a futuristic cabin that will make the entire air travel experience from home to gate to destination out of this world.

\begin{figure}[h]
  \centering
     \includegraphics[width=7cm]{images/image008.png}
   \caption{Scaled mock-up of seats on rails prototype}
  \label{fig:mock_up_on_rails}
\end{figure}


%\section*{Glossary}

%\begin{list}{-}{}

%\end{list}

% Set up the Glossary. The template is looking for a file called
% "glossaryterms.tex" with glossary terms and definitions.
% You can either edit this file manually or you
% can use the Memoir glossary feature in which you insert items like
%   \glossary{glossary term}{our definition of what the term means}
% wherever you like, as you write your documentation.
% When you run the report through Latex, it will create a ".glo" file like
% "OurFallDocument.glo" which you can edit to create the file "glossaryterms.tex"
% There is also perl script I made which will do the formatting for you. 
%  perl Glo2Tex OurFallDocument.glo > glossaryterms.tex

\newpage
\section*{Glossary}
\addcontentsline{toc}{section}{Glossary}
\label{sec-glossary}
\begin{description}
\item [3d audio technology] Simulation that creates the illusion of sound sources placed anywhere in 3 dimensional space, including behind, above or below the listener.
\item[action-event control] Process where a user action creates an physical event.
\item [API] Application Programming Interface.
\item[array of microphones] Microphones linked together to expand the effective coverage area. 
\item [Ausim] 3D audio hardware company.
\item [Automatic beam steering] Signal processing technique to narrow the microphone coverage area. Used to pick out a speaker and suppress background noise coming from directions other than that of the speaker.
\item[Benchmarking] A process of researching and observing to understand the state of the art for a given field or topic.
\item[Brainstorming] A process by which groups of people generate ideas
\item [Brainwaves] A common term that refers to post-synaptic potentials measured from many neurons in the brain
\item [CDR] Center for Design Research at Stanford University
\item [CFP] Otherwise known as a Critical Function Prototype, this is a prototype built to test a concept that is critical to addressing the problem statement.
\item [Client] Computer program that accesses a server.
\item [Client-server paradigm] A computing architecture which separates the client from a server over a computer network. 
\item [Crowded channel] A communication channel that is clogged with information.
\item [CVE] Acronym for Collaborative Virtual Environment. This is a virtual environment that support more than one user at the same time.
\item [Dark Horse] An idea that is unlike the others preceding it, an outlier.
   % input the list "glossaryterms.tex"
\end{description}

%%%%%% Example of an optionally printed "remark"
%\begin{remark}
%\color{blue}
%It's a sign of a successful team that the glossary becomes extensive. Define any non-obvious or invented terms. For %example, if you reference something by an acronym, that might be a glossary term. Teams also coin terms to describe %design features. Define such terms here.  Don't define obvious stuff (axle, keyboard).  

%See comments in me310report.tex if you want to generate a glossary semi-automatically from tagged keywords.
%\normalcolor
%\end{remark}

%%%%%%%%%%%%%%%%%%%%%%%%
% TOC and LOF are automatically generated -- Note that sometimes have to "compile" Latex THREE
% times to update the main .aux files, the TOC etc. files, and finally the PDF output with all changes
% propagated to the printout.
% Make Table of Contents title smaller than a normal Chapter heading:
\renewcommand{\chaptitlefont}{\normalfont\Large\bfseries}
\newpage
\tableofcontents %asterisk to prevent it from getting a number

% Optional Lists of Figures and Tables:
\newpage
\listoffigures  %Note that for this you probably want to add the [short-headings] to captions.
%\listoftables  %I decided to omit the LOT in this example.

%Back to normal size for subsequent sections
\renewcommand{\chaptitlefont}{\normalfont\Huge\bfseries}
%%%%%%%%%%%%%%%%%%%%%%%%

